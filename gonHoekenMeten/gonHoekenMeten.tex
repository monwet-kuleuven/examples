\documentclass{ximera}

\title[Examples:]{Inleiding goniometrie}

\begin{document}
\begin{abstract}
	Goniometrie is de studie van hoeken (en dus driehoeken, cosinussen, ...)
\end{abstract}
\maketitle


\subsection{Achtergrondinformatie}
	In het algemeen betekent 'iets meten' een maat (meestal een getal) associëren aan dat iets. We zullen dat hier doen voor hoeken.
\begin{expandable}
	In sommige gevallen is er een min of meer natuurlijke maateenheid. Om te meten hoe lang het duurt om wiskunde te leren ligt het enigszins voor de hand om het aantal 'dagen' (dus: omwentelingen van de aarde om haar as), 'maanden' (omwentelingen van de maan om de aarde) of 'jaren' (aantal omwentelingen van de aarde om de zon) te gebruiken. Voor kortere tijden hebben we sinds mensenheugnis enigszins arbitraire keuzes gemaakt: 'uren' zijn 1/24ste van een dag, en worden verdeeld in 60 'minuten' die op hun beurt verdeeld zijn in 60 'seconden'. Nog nauwkeuriger rekenen we toch weer decimaal: in tienden of honderdsten van een seconde. Voor afstanden waren er vroeger duimen, voeten, ellen, boogscheuten en dergelijke, maar is er (essentieel sinds Napoleon) de arbitraire, maar handige 'meter' met tiendelige verdelingen en veelvouden (cm, km, ...). Gelijkaardige situaties doen zich voorbij het meten van gewichten, krachten, elektrische ladingen enzovoort.

	Ook voor hoeken stelt zich het probleem van het 'meten' van zo'n hoek. Historisch -en tot vandaag in het dagelijkse leven- gebruikt men een systeem van graden, minuten en seconden. In de wetenschap worden hoeken echter bijna altijd gemeten in radialen. Dat is (veel) handiger.
\end{expandable}



\subsection{Inhoud}
\begin{expandable}
In deze module bestuderen we 
	\begin{itemize}
		\item het meten van hoeken in graden, minuten en seconden, maar in de wetenschap meestal in radialen.
        	\item de zogenaamde 'goniometrische getallen' van hoeken: sinus, cosinus en tangens
	    	\item een collectie eigenschappen en formules van hoeken en driehoeken: de Stelling van Pythagoras, cosinusregel, sinusregel etc.
        	\item de voorstelling op de goniometrische cirkel (met supplementaire, complementaire en tegengestelde hoeken)
	   	\item het oplossen van (al dan niet rechthoekige) driehoeken
	\end{itemize}

Het blijkt dat hoeken -misschien enigszins verrassend-  ook nauw verbonden zijn met de zogenaamde complexe getallen. Daarom besteden we in deze module ook daaraan aandacht. Complexe getallen worden ook gebruikt in andere delen van de wiskunde, zoals het oplossen van vergelijkingen, en in andere wetenschappen, bijvoorbeeld bij de studie van elektrische netwerken of trillingen.

We behandelen
	\begin{itemize}
		\item de definitie en voorstelling van complexe getallen, met de imaginaire eenheid i
		\item rekenen met complexe getallen en de complex toegevoegde $\bar{z}$ en norm $|z|$
		\item de exponentiële vorm $|z|e^{iθ}$ van complexe getallen
	\end{itemize}
\end{expandable}

\subsection{Wat moet je kennen en kunnen na deze module?}

\begin{expandable}
	\begin{itemize}
		\item Rekenen met hoeken in graden/minuten en seconden, en in radialen
		\item Je kent de sinus,cosinus en tangens van een hoek
		\item Je kent de hoofdformule van de goniometrie
	\end{itemize}
\end{expandable}

\subsection{Ken ik de leerstof al voldoende?}
\begin{expandable}
		  Druk uit in graden:
		  \begin{itemize}
	              \item 
			\begin{problem}
				\begin{hint} Een rechte hoek ... \end{hint}
				$\pi/2$: $\answer{90}^\circ$
			\end{problem}
	              \item 
			\begin{problem}
				\begin{hint} De helft van een rechte hoek ... \end{hint}
				$\pi/4$: $\answer{45}^\circ$
			\end{problem}
		      \item 
			\begin{problem}
				\begin{hint} $180^\circ = \pi$, dus $\dfrac{2\pi}{4} = ...$?\end{hint}
				$\dfrac{2\pi}{4}$: $\answer{120}^\circ$
			\end{problem}
		  \end{itemize}
\end{expandable}


\end{document}
