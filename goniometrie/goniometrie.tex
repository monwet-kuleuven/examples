\documentclass{ximera}

\title[Examples:]{Inleiding goniometrie}

\begin{document}
\begin{abstract}
	Wat is goniometrie? De studie van hoeken (en dus driehoeken, cosinussen, ...)
\end{abstract}
\maketitle

Wie technische wetenschappen studeert wordt voortdurend en soms onverwacht geconfronteerd met goniometrie en driehoeksmeetkunde. Van optica tot machinebouw, van elektrotechniek tot staalbouw, er is bijna geen vak of vakgebied te vinden waarbij goniometrie geen rol speelt. Hieronder staan enkele voorbeelden: een instrument uit de topografie en een toepassing uit de bouwkunde.

Een theodoliet is een instrument om hoeken te meten dat veel gebruikt wordt door landmeters. Het toestel wordt op een staander horizontaal (waterpas) geplaatst en door een kijker op verschillende referentiepunten te richten kan men de hoeken tussen deze punten meten.


\begin{expandable}
	Inleiding (Film, 
 \begin{center}
	 \youtube{2jGHOcxB8sI}
 \end{center}
\end{expandable}



Wat moet je kennen en kunnen na deze module?

\begin{foldable}
	\begin{itemize}
		\item Rekenen met hoeken in graden/minuten en seconden, en in radialen
		\item Je kent de sinus,cosinus en tangens van een hoek
		\item Je kent de hoofdformule van de goniometrie
	\end{itemize}
\end{foldable}


\end{document}
