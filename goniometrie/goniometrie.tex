\documentclass{ximera}

\title[Examples:]{Inleiding goniometrie}

\begin{document}
\begin{abstract}
	Wat is goniometrie? De studie van hoeken (en dus driehoeken, cosinussen, ...)
\end{abstract}
\maketitle

Wie technische wetenschappen studeert wordt voortdurend en soms onverwacht geconfronteerd met goniometrie en driehoeksmeetkunde. 

Van optica tot machinebouw, van elektrotechniek tot staalbouw, er is bijna geen vak of vakgebied te vinden waarbij goniometrie geen rol speelt. Hieronder staan enkele voorbeelden: een instrument uit de topografie en een toepassing uit de bouwkunde.

In een aantal gevallen blijkt het veel praktischer om hoeken te meten dan wel afstanden. Met een theodoliet kan dat bijvoorbeeld eenvoudig en nauwkeurig, en dat toestel wordt dan ook veel gebruikt door landmeters. Het toestel wordt op een staander horizontaal (waterpas) geplaatst en door een kijker op verschillende referentiepunten te richten kan men de hoeken tussen deze punten meten. Met behulp van goniometrische formules kunnen dan ook afstanden worden berekend.
\begin{image}
	\includegraphics{Askania_Sekunden-Theodolit_TU_e_400.jpg}
\end{image}

De staalconstructie van een hoogspanningsmast is gebaseerd op driehoeken. De studie van (de sterkte van) dergelijke complexe constructies is gebaseerd op de eigenqschappen van eenvoudige driehoeken.
\begin{image}
	\includegraphics{Pylon_ds.jpg}
\end{image}


\subsection{Inleiding (Video, 1:27)}
\begin{expandable}
 \begin{center}
	 \youtube{2jGHOcxB8sI}
 \end{center}
\end{expandable}



\subsection{Wat moet je kennen en kunnen na deze module?}

\begin{foldable}
	\begin{itemize}
		\item Rekenen met hoeken in graden/minuten en seconden, en in radialen
		\item Je kent de sinus,cosinus en tangens van een hoek
		\item Je kent de hoofdformule van de goniometrie
	\end{itemize}
\end{foldable}

\subsection{Ken ik de leerstof al voldoende?}
\begin{foldable}
		  Druk uit in graden:
		  \begin{itemize}
	              \item 
			\begin{problem}
				\begin{hint} Een rechte hoek ... \end{hint}
				$\pi/2$: $\answer{90}^\circ$
			\end{problem}
	              \item 
			\begin{problem}
				\begin{hint} De helft van een rechte hoek ... \end{hint}
				$\pi/4$: $\answer{45}^\circ$
			\end{problem}
		      \item 
			\begin{problem}
				\begin{hint} $180^\circ = \pi$, dus $\dfrac{2\pi}{4} = ...$?\end{hint}
				$\dfrac{2\pi}{4}$: $\answer{120}^\circ$
			\end{problem}
		  \end{itemize}
\end{foldable}


\end{document}
